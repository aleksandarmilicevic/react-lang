\section{Preliminaries}
\label{sec:prelim}

\subsection{Constructive Solid Geometry}

Constructive solid geometry (CSG) is a branch of geometry dedicated to building complex solid objects from a simple set of basic shapes and operators to modify and combine them.
%TODO some reference ?
As basic shapes we will use: Boxes, Cylinders, and Spheres.
%\item[Polytope]
The transformations that can be applied to objects are: rotations, translations, and scaling.
%TODO what about those other usefull stuff like moving the face perpendicular to the normal ?
Finally, shapes can be combined using the following boolean operators: union, intersection, difference.
%\item[Minowsky sum]
%\item[Convex hull]

%\paragraph{Joints.} In order to make our robots move, we add set of joint connectives 
%joints can be modelled in CSG, however it is good to have a library of them.

\subsection{Bond Graphs}
A bond graph~\cite{Paynter} is a graphical model for dynamic systems.
It expresses how energy is flows between different elements of the system.
Energy is represented by effort and flow which are linked by $\mathit{power} = \mathit{effort} \cdot \mathit{flow}$
Energy flows along the edges and nodes impose constraint on energy and flows. 

There are 5 types of on port nodes:
\begin{description}
\item[\bgR]  Resistance: $e = \bgR \cdot f$
\item[\bgC]  Capacitance: $\dot{e} = f/\bgC$
\item[\bgI]  Inertia: $\dot{f} = e/\bgI$
\item[\bgSe] Source of effort: $e = \bgSe$
\item[\bgSf] Source of flow: $f = \bgSf$
\end{description}

Devices like springs, torsion bars, and electrical capacitors are nodes of type C.
I nodes represent inductance in electrical systems and mass or inertia effects in mechanical systems.

There are 5 types of on port nodes:
\begin{description}
\item[\bgTF] Transformer: $f_1 \cdot r = f_2 \land e_1/r = e_2$
\item[\bgGY] Gyrator: $e_2 = \mu \cdot f_1 \land e_1 = \mu \cdot f_2$
\end{description}
    
An electric motor, in a bond graph, is a gyrator.
The output torque is proportional to the input current and the back electromotive force to the angular speed.


Finally, there are two kinds of junctions, or multiport nodes:
\begin{description}
\item[0 junction] the flow sums to zero and the efforts are equal: $\forall a,b.\, e_a = e_b \land \sum f = 0$
\item[1 junction] the efforts sum to zero and the flows are equal: $\forall a,b.\, f_a = f_b \land \sum e = 0$
\end{description}

Nodes and junctions are connected by edges which corresponds to equality of efforts and flows.

Graphs are multi-domains, i.e., electrical and mechanical energy can occur in the same graph.
Transformer nodes convert elements between domains.
For instance, an electric motor convert voltage and current int torque and angular speed.
    
In our context, we need the following energy domain:
\begin{description}
\item[Translation] where the effort is a force (\force) expressed in \unit{N} and the flow is velocity (\velocity) expressed in \unit{m \cdot s^{-1}}.
\item[Rotation] where the effort is a torque (\torque) expressed in \unit{N \cdot m} and the flow is angular velocity (\angularVelocity) expressed in \unit{rad \cdot s^{-1}}.
\item[Electric] where the effort is a electromotive force (\voltage) expressed in \unit{V} and the flow is current (\current) expressed in \unit{A}.
\end{description}

\subsection{Physics of solid objects}

\paragraph{Kinematic.}
The study of movement.

\paragraph{Dynamic.}
The study of how forces applied to the object generate movement.

