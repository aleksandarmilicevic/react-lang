\section{Introduction}
\label{sec:intro}

\paragraph{Domain of application.}
`Simple' robots which can easily be manufactured using DYI tools and off-the-shelf electronic.

\paragraph{Target audience.}
Students or prosumer hobbyists wanting to learn robotic.
We assume some basic technical/programming knowledge.
The user has control over the whole process: design, implementation, production, and test/deployment.
The language should simplify and unify those tasks.

If a program crashes hundreds of time on your computer it is just annoying.
On the other hand, if you crash hundreds of quadcopters you need a generous disposable income.

\paragraph{Use case / success scenario.}
Implementing/modeling a quadcopter such that:
\begin{itemize}
\item From the source we can generate (1) the code for the controller (ROS/Arduino) and (2) models for the parts that can be manufactured.
Vitamins (actuators, sensors, battery, and electronic) should be part of the description but are added by the user.
\item We can extract a kinetic+dynamic model from the source and simulate the robot (generating the required files for a simulator like gazebo).
\item Prove some safety properties about the robot, e.g., assuming it has distance sensors, it will not let the user smash it into a wall.
\end{itemize}
Obviously a quadcopter is pretty challenging, so we are going to start with a simple 3 wheeled robot.
Two fixed wheels (left/right) with the same axis of rotation and each connected to a motor and a castor wheel (tail) for stabilisation.

The success is more killer app, rather than a killer technique.
Most of the elements exist on in one form or another.
However, to the best of my knowledge, they have not been put together is such a way.

\paragraph{Contributions.}
\begin{itemize}
\item Rethinking the stack for designing robots.
\item Bringing the physical world in the control logic.
\item Simulation / verification of the robot before it reaches production. (later synthesis...)
\end{itemize}

There are CAD tools to design the robot.
There are analysis tools for dynamic systems.
However, the two are separated and one ends up building twice the same robot into two different systems.
Furthermore, programming is a separated task and ones build a (third) logical representation of the robot.
This separation might make sense for large teams working on complex robots.
However, for our target user there is a lot of repetition and overlap.

\paragraph{Possible collaborations.}
There is a lot to do and we want to reuse as much as possible (especially about the case study) from what the other teams in the PPR projects are doing.
\begin{itemize}
\item A team at MIT+UPenn is trying to extract dynamic models from physical description (proving that a design performs as expected). This is very related to what we are trying to do. Good chances of collaboration.
\item A team at UPenn is trying to print a quadcopter (including part that usually considered as vitamin such as the wires). No direct collaboration but can be a valuable example/inspiration for the case study.
\end{itemize}

