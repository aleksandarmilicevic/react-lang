\documentclass{article}

\usepackage[utf8]{inputenc}
\usepackage{amssymb,amsmath,amsfonts}
\usepackage{url}
\usepackage{mathpartir}

\title{Interface for ROSlab and REACT}
\date{\today}

\begin{document}

\maketitle

\newcommand{\tr}[1]{\stackrel{#1}{\longrightarrow}}

For the segway robot: so we give a 2 dimensional model.
%assume the center of the robot's frame is the geometric center of the robot.\\
This model is obviously (too simple) and idealized but it gives us a starting point.

REACT will provide a global controller that sends commands to the robots.
The robots store the commands in a local FIFO buffer until they executed.
The commands tells the robot how to move.
In this version, the controller tells the robot at which speed to move and rotate.
Furthermore, we assume the robots have a clock (assumed perfect), the message receptions (and similar computations) take no time.

If everybody agrees with the semantics, we can proceed to specify a set of ROS messages for the integration of the REACT controller in a system of robots programmed with ROSlab.
We can also discuss other types of commands, e.g., trajectory/path to follow, gripping.

\paragraph{Robot description.}~\\

A robot is described by the tuple $(id, \vec p, \theta, \mathit{cmds})$ where
\begin{itemize}
\item $id$ is a unique identifier.
\item $\vec p$ is the position. In the 2D case, $\vec p = \begin{pmatrix} x \\ y \end{pmatrix}$.
\item $\theta$ is the angle between the robot frame and the global frame. In the 2D case, it is a single angle.
\item $\mathit{cmds}$ is a queue of commands. We use $c::C$ to match command $c$ at the head of the queue $C$ and $C::c$ for the tail. $[]$ is the empty queue.
\end{itemize}

Each command has the form ``$\text{move}(v, \omega, t)$'' where
$v$ is the speed (along the $x$ axis of the robot frame),
$\omega$ is an angular speed (rotation center is $(0,0)^T$ in the robot frame),
$t$ is the time for witch this command should be executed.

\paragraph{Semantic for a single robot.}~\\

A transition from state $R$ to $R'$ taking time $t$ and with action $a$ is written as $R \tr{a,t} R'$.
The action $a$ is used to synchronize messages.
$\tau$ is the silent action.
$id?c$ means robot with identifier $id$ receive a message.
$id!c$ means a message is sent to robot with identifier $id$.

\[
\inferrule[idle]
          { R = (id, \vec p, \theta, []) \\ dt \geq 0}
          { R \tr{\tau,dt} R }
\]

\[
\inferrule[recv]
          { R = (id, \vec p, \theta, C) \\
            R' = (id, \vec p, \theta, C :: c) }
          { R \tr{id?c,0} R' }
\]

\[
\inferrule[cmd done]
          { R = (id, \vec p, \theta, \text{move}(v, \omega, 0) :: C) \\
            R' = (id, \vec p, \theta, C) }
          { R \tr{\tau,0} R' }
\]

\[
\inferrule[exec]
    { R = (id, \vec p, \theta, \text{move}(v, \omega, t) :: C) \\
      R' = (id, \vec p', \theta', \text{move}(v, \omega, t - dt) :: C ) \\\\
      t \geq dt \geq 0 \\
      \theta' = \theta + \omega \cdot dt \\
      \vec p' = \vec p + \left({\begin{tabular}{cc} $\cos(\theta)$ & $\sin(\theta)$ \\ $\sin(\theta)$ & $\cos(\theta)$ \end{tabular}}\right) \vec q \\\\
      \text{where } \vec q = \left\{ {\begin{tabular}{lr} $(v \cdot dt, 0)$ & if $\omega = 0$ \\
                                     $(v/\omega \cdot \sin(\omega \cdot dt), v/\omega \cdot (1- \cos(\omega \cdot dt)))$ & otherwise \end{tabular}} \right.\\
    }
    { R \tr{\tau,dt} R' }
\]

\paragraph{Multiple robots.}

We have transitions of the individual robots are labeled with time.
When synchronising the transitions of multiple robots we simply requires that the time on the transition of every robot agree:

\[
\inferrule[time]
          { \forall i \in I. R_i \tr{\tau,t} R_i' }
          { \prod_{i \in I} R_i \tr{\tau,t} R_i' }
\]

\noindent
$||$ denotes the parallel composition, $\prod$ the index parallel composition.

For messages, we require:
\[
\inferrule[msg]
          { R_a \tr{id!c,0} R_a' \\ R_b \tr{id?c,0} R_b' \\
            \forall i \in I \setminus \{a,b\}. R_i \tr{\tau,0} R_i' }
          { R_a || R_b ~|| \prod_{i \in I \setminus \{a,b\}} R_i ~~\tr{\tau,0}~~ R_a' || R_b' ~|| \prod_{i \in I \setminus \{a,b\}} R_i' }
\]

In this formulation the controller is a special robot with the ability to send messages.

\end{document}
